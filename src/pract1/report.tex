\graphicspath{{/img}} % path to graphics

\section*{\LARGE Введение}
\addcontentsline{toc}{section}{Введение}

Apache Cassandra --- открытая, распределенная, децентрализованная,
эластично масштабируемая, высокодоступная, отказоустойчивая, строковая,
допускающая настройку согласованности база данных, дизайн
распределенности которой основан на Amazon Dynamo, а модель данных - на
Google Bigtable.
Разработана для социальной сети, ныне используется на
некоторых из наиболее популярных веб-сайтах.

\textbf{Цель работы} --- познакомится с СУБД Apache Cassandra.

\clearpage

\section*{\LARGE Выполнение практической работы}
\addcontentsline{toc}{section}{Выполнение практической работы}

\section{Создание Docker контейнера}

При первоначальном знакомстве с СУБД предлагается не углубляться в
её распределенную природу, а научиться работать со средствами,
позволяющими получить быстрый доступ к её основным функциям на одном
узле. Речь ведётся о средствах контейнеризации Docker. Для этого
предлагается пройти регистрацию и установить Docker Desktop на ПК.\par
Далее необходимо запустить контейнер с Cassandra (рис.~\ref{fig:docker}).

\begin{enumerate}
	\item Введите в терминале команду:\\
		\verb|docker run -d –name [имя контейнера][образ]|
	\item Проверьте работоспособность контейнера командой:\\
		\verb|docker ps|
	\item Чтобы взаимодействовать с СУБД, необходимо получить доступ в
		оболочку для работы с CQL – языком запросов в Cassandra. Для этого
		введите команду:\\
		\verb|docker exec -ti [имя контейнера] cqlsh|
	\item Для остановки и перезапуска контейнера используйте:\\
		\verb|docker stop [имя контейнера]|\\
		\verb|docker start [имя контейнера]|
	\item Для удаления контейнера используйте:\\
		\verb|docker rm [имя контейнера]|
\end{enumerate}

\begin{image}
	\includegrph{img/img1.png}
	\caption{Запуск Docker контейнера}
	\label{fig:docker}
\end{image}

\section{Простые команды cqlsh}
\subsection{Справка}

Для получения справки по cqlsh введите \verb|HELP| или \verb|?;|
будут выведены все существующие команды (рис.~\ref{fig:help}).

\begin{image}
	\includegrph{img/img2.png}
	\caption{Ввод комнды HELP}
	\label{fig:help}
\end{image}

\subsection{Описание окружения в cqlsh}

Если вы пользуетесь двоичным дистрибутивом, то после подключения
к экземпляру Test Cluster создается пустое пространство ключей, т. е. база
данных Cassandra. С этой базой можно экспериментировать.\par
Чтобы узнать о текущем кластере, в котором вы работаете, введите:\\
\verb|cqlsh> DESCRIBE CLUSTER;|
Чтобы узнать, какие пространства ключей имеются в кластере, наберите:\\
\verb|cqlsh> DESCRIBE KEYSPACES;|

\begin{image}
	\includegrph{img/img3.png}
	\caption{Окружения в cqlsh}
	\label{fig:keyspaces}
\end{image}

\subsection{Создание пространства ключей и таблицы в cqlsh}

Пространство ключей в Cassandra напоминает реляционную базу данных. В
нем определяется одна или несколько таблиц, или "<семейств столбцов"> Если
запустить cqlsh, не указав пространство ключей, то будет напечатано
приглашение \texttt{cqlsh>}, в котором имя пространства ключей отсутствует.\par
Создадим свое пространство ключей, чтобы было куда записывать данные.
\begin{verbatim}
CREATE KEYSPACE my_keyspace WITH replication = {
	'class': 'SimpleStrategy'
	, 'replication_factor': 1
	};
\end{verbatim}

Создав пространство ключей, мы можем переключиться на него:
\begin{verbatim}
cqlsh> USE my_keyspace;
cqlsh:my_keyspace>
\end{verbatim}

Приглашение изменилось --- теперь в него входит имя пространства ключей.\par
Имея пространство ключей, мы можем создать в нем таблицу. В cqlsh для
этого служит такая команда:

\begin{verbatim}
cqlsh:my_keyspace> CREATE TABLE user (
	first_name text
	, last_name text
	, PRIМARY КЕУ (first_name)
	);
\end{verbatim}

В результате в текущем пространстве ключей создается таблица "<user"> с
двумя столбцами для хранения имени и фамилии, оба типа text. Типы text и
varchar -- синонимы и применяются для хранения строк. Мы сказали, что
столбец \textit{first\_name} будет первичным ключом, а для всех остальных
параметров таблицы оставили значения по умолчанию.\par
С помощью команды \texttt{DESCRIBE TABLES} можно получить описание только
что созданной таблицы (рис.~\ref{fig::create_table}).

\begin{image}
	\includegrph{img/img4.png}
	\caption{Создание таблицы}
	\label{fig::create_table}
\end{image}

\subsection{Запись и чтение данных в cqlsh}
Для записи значения служит команда INSERT:

\begin{verbatim}
cqlsh:my_keyspace> INSERT INTO user (first_name, last_name) VALUES
('Bill', 'Nguyen');
\end{verbatim}

Прочитаем их командой SELECT:

\begin{verbatim}
cqlsh:my_keyspace> SELECT * FROM user WHERE first_name='Bill';
\end{verbatim}

Команда DELETE позволяет удалить столбец. Ниже из строки с ключом Bill
удаляется столбец \textit{last\_name}:

\begin{verbatim}
cqlsh:my_keyspace> DELETE last_name FROM user WHERE first_name='Bill';
\end{verbatim}

Убедимся, что столбец действительно удален:

\begin{verbatim}
cqlsh:my_keyspace> SELECT * FROM user WHERE first_name='Bill';
\end{verbatim}

Теперь почистим за собой, удалив всю строку. Команда та же самая, только
имя столбца не указывается:

\begin{verbatim}
cqlsh:my_keyspace> DELETE FROM user WHERE first_name='Bill';
\end{verbatim}

Убедимся, что строка удалена:

\begin{verbatim}
cqlsh:my_keyspace> SELECT * FROM user WHERE first_name='Bill';
\end{verbatim}

Можно удалить из таблицы все данные командой TRUNCATE или вообще
удалить схему таблицы командой DROP TABLE:

\begin{verbatim}
cqlsh:my_keyspace> TRUNCATE user;
cqlsh:my_keyspace> DROP TABLE user;
\end{verbatim}

\begin{image}
	\includegrph{img/img6.png}
	\caption{Удаление через TRUNCATE}
	\label{fig::TRUNCATE}
\end{image}


\begin{image}
	\includegrph{img/img5.png}
	\caption{Запись и чтение данных}
	\label{fig::fill}
\end{image}


\clearpage

\section*{\LARGE Вывод}
\addcontentsline{toc}{section}{Вывод}
В результате практической работы была изучена работа
с СУБД Apache Cassandra.\par
Научились запускать образ СУБД в контейнере.
Изучили, что такаое пространство ключей.
Узанали, как создавать таблицы, как их заполнять, получать из них данный.
А также удалять данный из таблиц.

