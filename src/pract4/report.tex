\graphicspath{{/img}} % path to graphics

\section*{\LARGE Выполнение практической работы}
\addcontentsline{toc}{section}{Выполнение практической работы}

\section{Создание кластера на 3 узла с помощью Docker Compose}

В первой части практики необходимо составить docker-compose.yml.
Требования, предъявляемые к кластеру:
\begin{itemize}
	\item 3 сервиса: cassandra-seed-shectakov, cassandra-node-shectakov\_1, cassandra-node-shectakov\_2
	\item Для каждого сервиса изменить (использовать) следующие
	стандартные переменные в environment:
	\begin{itemize}
		\item CASSANDRA\_CLUSTER\_NAME
		\item CASSANDRA\_ENDPOINT\_SNITCH
		\item CASSANDRA\_DC
	\end{itemize}
	\item Узлы cassandra-seed-\ldots и cassandra-node-\ldots1 должны
	находиться в одинаковых датацентрах
	\item Для cassandra-node-\ldots1 и cassandra-node-\ldots2 добавить
	настройку depends\_on в значении которой указать cassandra-seed-\ldots
	\item Создать собственный volume для каждого сервиса чтобы было проще
	искать данные (пример \("\)cassandra\_data\_seed:/var/lib/cassandra\("\))
	\item Открыть порты для каждого сервиса (пример 19042:9042)
	\item Для главного узла cassandra-seed-*фамилия студента*
	необходимо написать в environment AUTO\_BOOTSTRAP=false
\end{itemize}

\lstinputlisting[language=yaml]{src/docker-compose.yml}

Запустим его:

\begin{image}
	\includegrph{img/img1.png}
	\caption{Запуск Docker compouse}
	\label{fig:docker}
\end{image}

Для того чтобы удостовериться, что весь кластер работает правильно
необходимо выполнить команду \texttt{docker exec -ti cas1 nodetool status}
(рис.~\ref{fig:status}).

\begin{image}
	\includegrph{img/img2.png}
	\caption{Вывод статуса узлов}
	\label{fig:status}
\end{image}

Теперь необходимо создать свою схему бд (в кассандре это
пространство ключей) для этого необходимо перейти в утилиту \texttt{cqlsh}
при помощи команды \texttt{docker exec -ti cas1 cqlsh} и выполнить скрипт,
где название кейспейса можно придумать самому (желательно номер
студенческого или другой идентификатор).

\begin{image}
	\includegrph{img/img3.png}
	\caption{Создание пространства ключей}
	\label{fig:keyspace}
\end{image}

Создадим базу данных и заполним ее значениями.
(рис.~\ref{fig:full}).

\begin{image}
	\includegrph{img/img4.png}
	\caption{Заполнение таблицы}
	\label{fig:full}
\end{image}

Теперь проверим, что на всех узлах есть созданные таблицы
(рис.~\ref{fig:check}).

\begin{image}
	\includegrph{img/img5.png}
	\caption{Содержимое пространства ключей на разных узлах}
	\label{fig:check}
\end{image}

\clearpage

Модель базы данных из прошлого семестра (рис.~\ref{fig:past}).

\begin{image}
	\includegrph{img/img6.png}
	\caption{Модель базы данных из прошлого семестра}
	\label{fig:past}
\end{image}

\section*{\LARGE Вывод}
\addcontentsline{toc}{section}{Вывод}
В результате практической работы была изучена работа
с распределенной бд, используя Apache Cassandra.\par
Были созданы докер контейнеры прдеставляющие разные узлы.
И базе данных созданы таблицы.