\graphicspath{{img}} % path to graphics

\section*{\LARGE Выполнение практической работы}
\addcontentsline{toc}{section}{Выполнение практической работы}

\section{Создание кластера на 3 узла с помощью Docker Compose}

В первой части практики необходимо составить docker-compose.yml.
Требования, предъявляемые к кластеру:
\begin{itemize}
	\item 3 сервиса: cassandra-seed-shectakov, cassandra-node-shectakov\_1, cassandra-node-shectakov\_2
	\item Для каждого сервиса изменить (использовать) следующие
	стандартные переменные в environment:
	\begin{itemize}
		\item CASSANDRA\_CLUSTER\_NAME
		\item CASSANDRA\_ENDPOINT\_SNITCH
		\item CASSANDRA\_DC
	\end{itemize}
	\item Узлы cassandra-seed-\ldots и cassandra-node-\ldots1 должны
	находиться в одинаковых датацентрах
	\item Для cassandra-node-\ldots1 и cassandra-node-\ldots2 добавить
	настройку depends\_on в значении которой указать cassandra-seed-\ldots
	\item Создать собственный volume для каждого сервиса чтобы было проще
	искать данные (пример \("\)cassandra\_data\_seed:/var/lib/cassandra\("\))
	\item Открыть порты для каждого сервиса (пример 19042:9042)
	\item Для главного узла cassandra-seed-*фамилия студента*
	необходимо написать в environment AUTO\_BOOTSTRAP=false
\end{itemize}

\lstinputlisting[language=yaml]{src/docker-compose.yml}

Запустим его:

\img{img.png}{Запуск Docker compouse}

\section{Ключи в Cassandra}

Для начала подключимся к любому из контейнеру.
\begin{verbatim}
	docker exec -it cas1 cqlsh
\end{verbatim}

\begin{verbatim}
	CREATE KEYSPACE learn_cassandra WITH REPLICATION = {
	'class' : 'NetworkTopologyStrategy',
	'datacenter1' : 2
	};
\end{verbatim}

\img{img1.png}{Содание KEYSPACE}

Создадим таблицу
\begin{verbatim}
CREATE TABLE learn_cassandra.users_by_country (
	country text,
	user_email text,
	first_name text,
	last_name text,
	age smallint,
	PRIMARY KEY ((country), user_email)
);
\end{verbatim}

Давайте наполним таблицу некоторыми данными
\begin{verbatim}
INSERT INTO learn_cassandra.users_by_country
(country,user_email,first_name,last_name,age)
 VALUES('US', 'john@email.com', 'John','Wick',55);
INSERT INTO learn_cassandra.users_by_country
(country,user_email,first_name,last_name,age)
 VALUES('UK', 'peter@email.com', 'Peter','Clark',65);
INSERT INTO learn_cassandra.users_by_country
(country,user_email,first_name,last_name,age)
 VALUES('UK', 'bob@email.com', 'Bob','Sandler',23);
INSERT INTO learn_cassandra.users_by_country
(country,user_email,first_name,last_name,age)
 VALUES('UK', 'alice@email.com', 'Alice','Brown',26);
\end{verbatim}

Давайте создадим еще одну таблицу и заполним их записями.
Раздел будет определяться только столбцом user\_email.

\begin{verbatim}
CREATE TABLE learn_cassandra.users_by_email (
 user_email text,
 country text,
 first_name text,
 last_name text,
 age smallint,
 PRIMARY KEY (user_email)
);
INSERT INTO learn_cassandra.users_by_email (user_email,
country,first_name,last_name,age)
 VALUES('john@email.com', 'US', 'John','Wick',55);
INSERT INTO learn_cassandra.users_by_email
(user_email,country,first_name,last_name,age)
 VALUES('peter@email.com', 'UK', 'Peter','Clark',65);
INSERT INTO learn_cassandra.users_by_email
(user_email,country,first_name,last_name,age)
 VALUES('bob@email.com', 'UK', 'Bob','Sandler',23);
INSERT INTO learn_cassandra.users_by_email
(user_email,country,first_name,last_name,age)
 VALUES('alice@email.com', 'UK', 'Alice','Brown',26);
\end{verbatim}

Кассандра старается избегать вредных запросов.
Если вы хотите выполнить фильтрацию по столбцу, который не является ключом раздела, вам
необходимо это явно указать, как показано в следующей команде:

\begin{verbatim}
SELECT * FROM learn_cassandra.users_by_email WHERE age=26 ALLOW FILTERING;
\end{verbatim}

\img{img2.png}{Фильтрация по столбцу}

\section{Согласованность}

Мы устанавливали коэффициент репликации равным 2.
Таким образом, можно использовать уровень согласованности ALL или TWO.
Для выбора уровня согласованности пропишем команды, а дальше получим данные из БД.

\begin{verbatim}
CONSISTENCY ALL;
SELECT * FROM learn_cassandra.users_by_country WHERE country='US';
\end{verbatim}

Если высокая согласованность не важна, то можно прописать уровень
согласованности, равный 1.
За счет этого повышается скорость обработки запросов

\begin{verbatim}
CONSISTENCY ONE;
SELECT * FROM learn_cassandra.users_by_country WHERE country='US';
\end{verbatim}

\img{img3.png}{Согласованность}

По умолчанию в таблицах используется SizeTieredCompactionStrategy.
Проверим это командой
\begin{verbatim}
DESCRIBE TABLE learn_cassandra.users_by_country;
\end{verbatim}



\img{img4.png}{users\_by\_country}

В таблице users\_by\_country можно определить age как еще один столбец
кластеризации для сортировки данных.
\begin{verbatim}
CREATE TABLE learn_cassandra.users_by_country_sorted_by_age_asc (
 country text,
 user_email text,
 first_name text,
 last_name text,
 age smallint,
 PRIMARY KEY ((country), age, user_email)) WITH CLUSTERING ORDER BY (age ASC);
\end{verbatim}

Вставьте данные в таблицу и удостоверьтесь, что записи придут в порядке
увеличения возраста

\begin{verbatim}
INSERT INTO learn_cassandra.users_by_country_sorted_by_age_asc
(country,user_email,first_name,last_name,age) VALUES('US','john@email.com',
'John','Wick',10);
INSERT INTO learn_cassandra.users_by_country_sorted_by_age_asc
(country,user_email,first_name,last_name,age) VALUES('UK', 'peter@email.com',
'Peter','Clark',30);
INSERT INTO learn_cassandra.users_by_country_sorted_by_age_asc
(country,user_email,first_name,last_name,age) VALUES('UK', 'bob@email.com',
'Bob','Sandler',20);
INSERT INTO learn_cassandra.users_by_country_sorted_by_age_asc
(country,user_email,first_name,last_name,age) VALUES('UK', 'alice@email.com',
'Alice','Brown',40);
SELECT * FROM learn_cassandra.users_by_country_sorted_by_age_asc;
\end{verbatim}

\img{img5.png}{users\_by\_country\_sorted\_by\_age\_asc}


\section*{\LARGE Вывод}
\addcontentsline{toc}{section}{Вывод}
В результате практической работы были изучены Особенности при создании таблиц Cassandra, ключи и согласованность
в Cassandra.