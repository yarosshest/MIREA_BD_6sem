\section*{\LARGE Введение}
\addcontentsline{toc}{section}{Введение}
\section{Особенности при создании таблиц Cassandra}
ОТСУТСТВИЕ JOIN \par
В Cassandra нет опции join. Чаще всего, это ограничение обходится с
помощью денормализации данных по дополнительным таблицам.

ОТСУТСТВИЕ ВНЕШНИХ КЛЮЧЕЙ \par
В Cassandra нет способа связывать сущности из разных таблиц между
собой, как например, это сделано в реляционных базах с помощью внешних
ключей. Как следствие отсутствуют такие операции как каскадное удаление.

ДЕНОРМАЛИЗАЦИЯ \par
Сама по себе денормализация обладает одним достоинством, в отличие от
обычного подход. Ее удобно использовать для хранения исторических данных,
которые ни при каких обстоятельствах не должны изменяться. В качестве
примера можно привести хранение покупок клиента со ссылками на товар. Т.к.
цена товара меняется с течением времени, чтобы получить стоимость каждого
товара на момент приобретения, нужно отдельно хранить цену товара. При
использовании подхода с денормализацией, можно в таблице с покупками
хранить полностью объект товара. При этом текущий товар можно изменять
любым образом и даже удалять и это никак не повлияет на историю покупок
клиента

ЗАПРОС ПЕРВИЧЕН \par
При проектировании, нужно учитывать, что запрос должен получить за
один раз все данные из одной таблицы (т.к. нет join). Поэтому сначала
проектируются все возможные запросы, а затем под них создаются таблицы.
Таким образом в Cassandra создание структуры БД начинается с определения
запросов.

ОПТИМАЛЬНОЕ ХРАНЕНИЕ \par
В реляционных базах редко можно встретить рекомендации к структуре
БД для оптимального хранения и чтения данных. Чаще всего никто не заботится
об этом. В распределенных системах дело обстоит иначе, т.к. данные
располагаются на нескольких узлах, более высокую производительно будет
демонстрировать тот запрос, который отдает данные с одной ноды. Таким
образом, желательно располагать данные так, чтобы данные возвращались из
одной ноды.

\clearpage
