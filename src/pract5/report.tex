\graphicspath{{img}} % path to graphics
\section*{\LARGE Введение}
\addcontentsline{toc}{section}{Введение}

Данные записываются в Cassandra таким образом, чтобы обеспечить
полную надежность и высокую производительность. Но в связи с тем, что данная
СУБД является распределенной, уходит некоторое время на обработку запросов
и передачу информации между узлами. Время обработки запросов называется
задержкой. Причинами больших "<задержек записи"> или \textit{"write latency"}
(при выполнении операций вставки и обновления) могут быть следующие факторы:

\begin{itemize}
	\item Низкая скорость записи дисков, на которых расположена СУБД;
	\item Большой размер батча;
	\item Запись строки большого размера. В идеале размер строки не должен
	превышать 1 мегабайта;
	\item Сетевая задержка, когда пропускная способность сети низка;
	\item Отсутствие свободных потоков для записи;
	\item Загруженность центрального процессора;
	\item Работа «сборщика мусора» (garbage collector).
\end{itemize}

Что касается "<задержки чтения"> или \textit{"read latency"},
то тут гораздо больше возможных причин ее возникновения:

\begin{itemize}
	\item Большой размер раздела (partition). При создании таблиц в
	Cassandra важно уметь создать правильный ключ раздела, чтобы разделы не
	были слишком большими;
	\item Плохой запрос. Сюда входит полное сканирование таблицы, поиск
	по ключу, который не является ключом раздела (ALLOW FILTERING);
	\item Количество затронутых sstable;
	\item Использование в качестве ключей кластеризации слишком
	уникальных (почта, номер телефона) или полей с малым количеством
	возможных значений (пол);
	\item Уровень согласованности. При большом уровне необходимо
	дождаться ответа большего количества узлов;
	\item Большое количество tombstone;
	\item Большое количество записей в секунду.
\end{itemize}

\section*{\LARGE Выполнение практической работы}
\addcontentsline{toc}{section}{Выполнение практической работы}

\section{Read latency и write latency}

В данной практической работе работе будет рассматриваться система с
открытым исходным кодом \textit{"Grafana"}, которая предоставляет функции
аналитики и визуализации данных, которые получаются из настраиваемых
источников и отображаются на панели (dashboard). Получать данные и хранить
будет бесплатное ПО под названием \textit{"Prometheus"}. Использовать для
визуализации данные статистики будет Grafana. В дальнейшем Grafana может
помочь вам визуализировать не только данные, полученные из Cassandra, но и
многих других источников.\par
Необходимо скачать созданный преподавателем репозиторий
и запустить 2 кластера, расположенных в 2 каталогах, с
помощью команды, указанной ниже. Команду необходимо выполнить в каждой
папке.

\begin{lstlisting}[language=bash]
docker-compose up -d
\end{lstlisting}

\section{Настройка Grafana}

Web-интерфейс Grafana расположен по адресу \texttt{localhost:3000}.
Для входа необходимо использовать логин и пароль \textit{"admin"}.
После этого нам необходимо добавить панель (dashboard).
Для этого наводим на плюс в левой части экрана, выбираем \textit{"Import"}.\par
Для данной практической работы будет использована панель
\textit{"Cassandra Dashboard"}, взятая из открытых источников.
Для импорта панели необходимо взять ID панели (в данном случае 12086)
и вставить в поле для импорта.

\begin{image}
	\includegrph{Screenshot from 2024-03-23 12-25-35}
	\caption{Cassandra Dashboard}
	\label{fig:dashboard:import}
\end{image}

Далее указываем в качестве источника Prometheus, а в интервале
обновления прописываем \textit{"15s"}, после чего производим импорт панели.\par
В данной панели пользователь может установить необходимые фильтры
(необходимые датацентр, keyspace и таблица). Так как пока ничего не создано,
можно выбрать только keyspace \textit{"system"}.

\begin{image}
	\includegrph{Screenshot from 2024-03-23 12-35-03}
	\caption{Настройка Dashboard}
	\label{fig:dashboard:setting}
\end{image}

\section{Создание таблиц}

Для начала нам необходимо создать таблицу, над которой мы будем
производить манипуляции.

\begin{image}
	\includegrph{Screenshot from 2024-03-23 12-37-30}
	\caption{Создание таблиц}
	\label{fig:keyspace}
\end{image}

После создания необходимо обновить страницу Grafana. Там
автоматически будет выбран созданный keyspace.

\section{Задержка при записи (Write latency)}

Вам необходимо подключиться к кластеру, как это производилось в
Практической работе \No1, и с помощью средств языка программирования Python
реализовать однопоточную и многопоточную вставку множества данных в
таблицу.

\begin{lstlisting}[language=Python]
#!./venv/bin/python3

from cassandra.cluster import Cluster
from cassandra.auth import PlainTextAuthProvider
from concurrent.futures import ThreadPoolExecutor
from faker import Faker
from time import sleep

fake = Faker()
contact_points = ['0.0.0.0']
cassandra_port = 9100
keyspace = 'learn_cassandra'
table_name = 'users_by_country'

cluster = Cluster(contact_points=contact_points)
session = cluster.connect(keyspace)

def insert_data(session):
    insert_query = f"INSERT INTO {table_name}"
    insert_query += " (country, user_email, first_name, last_name, age)"
    insert_query += " VALUES"
    country = fake.country()
    country = country.replace("'", "_")
    insert_query += f" (\'{country}\', \'{fake.email()}\'"
    insert_query += f", \'{fake.first_name()}\', \'{fake.last_name()}\'"
    insert_query += f", {fake.random_int(min=18, max=90)});"
    session.execute(insert_query)

def single_threaded_insert(session, num_rows):
    for _ in range(num_rows):
        insert_data(session)

def multi_threaded_insert(session, num_rows, num_threads):
    with ThreadPoolExecutor(max_workers=num_threads) as executor:
        for _ in range(num_rows):
            executor.submit(insert_data, session)

if __name__ == "__main__":
    num_rows = 100
    num_threads = 4

    single_threaded_insert(session, num_rows)

    sleep(10)

    multi_threaded_insert(session, num_rows, num_threads)

    session.shutdown()
    cluster.shutdown()
\end{lstlisting}


После записи вам необходимо узнать количество вставок в секунду и
среднюю задержку, найдя графики Writes, Write Latency (95\%) и Write Latency
(99\%) и отобразив это в отчете.\par
Latency 95\% и 99\% указывают вам точку, в которой 95\% и 99\% вашего
трафика испытывают задержку, меньшую, чем эти значения. Или, что еще более
важно, это значит, что 5\% и 1\% вашего трафика испытывают задержку вне этого
диапазона.

\begin{image}
	\includegrph{write}
	\caption{График Write}
	\label{fig:writing}
\end{image}

\begin{image}
	\includegrph{write_l95}
	\caption{График Write Latency 95\%}
	\label{fig:write_l95}
\end{image}

\begin{image}
	\includegrph{write_l99}
	\caption{График Write Latency 99\%}
	\label{fig:write_l99}
\end{image}

\section{Задержка при чтении (Read Latency)}

Вам необходимо подключиться к кластеру и реализовать однопоточное и
многопоточное чтения данных из таблицы. Изучите также зависимость задержки
от количества потоков: 10, 25, 50, 100.\par
После записи вам необходимо узнать количество чтений в секунду и
среднюю задержку, найдя графики Reads, Read Latency (95\%) и Read Latency
(99\%) и отобразив это в отчете. Для проверки задержки чтения используйте
следующие заготовки. Возраст и страну укажите сами в соответствии со
вставленными данными.

Перпишем скрипт для записи на чтение:

\begin{lstlisting}[language=Python]
#!./venv/bin/python3

from cassandra.cluster import Cluster
from cassandra.auth import PlainTextAuthProvider
from concurrent.futures import ThreadPoolExecutor
from faker import Faker
from time import sleep

fake = Faker()
contact_points = ['0.0.0.0']
cassandra_port = 9100
keyspace = 'learn_cassandra'
table_name = 'users_by_country'

cluster = Cluster(contact_points=contact_points)
session = cluster.connect(keyspace)

def single_threaded_read(session, query):
    return session.execute(query)

def multi_threaded_read(session, queries, num_threads):
    with ThreadPoolExecutor(max_workers=num_threads) as executor:
        for query in queries:
            result = executor.submit(query)

if __name__ == "__main__":
    num_rows = 100
    num_threads = 4

    queries = [
        "SELECT * FROM learn_cassandra.users_by_country;",
        "SELECT * FROM learn_cassandra.users_by_country WHERE age=30 ALLOW FILTERING;",
        "SELECT * FROM learn_cassandra.users_by_country WHERE country='USA';"
    ]
    thread_counts = [10, 25, 50, 100]

    for query in queries:
        print(single_threaded_read(session, query))

	sleep(10)

    for thread_count in thread_counts:
        print(f"read with {thread_count} thread")
        multi_threaded_read(session, queries, thread_count)

    session.shutdown()
    cluster.shutdown()
\end{lstlisting}

\begin{image}
	\includegrph{read}
	\caption{График Read}
	\label{fig:reading}
\end{image}

\begin{image}
	\includegrph{read_l99}
	\caption{График Read Latency 99\%}
	\label{fig:read_l99}
\end{image}

\begin{image}
	\includegrph{read_l95}
	\caption{График Read Latency 95\%}
	\label{fig:read_l95}
\end{image}

\clearpage

\section*{\LARGE Вывод}
\addcontentsline{toc}{section}{Вывод}
В данной практической работе мы узнали, что в СУБД Cassandra,
так как она является распределенной,
уходит некоторое время на обработку запросов
и передачу информации между узлами.
Время обработки запросов называется задержкой