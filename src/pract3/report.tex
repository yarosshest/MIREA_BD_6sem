\graphicspath{{/img}} % path to graphics


\section*{\LARGE Выполнение практической работы}
\addcontentsline{toc}{section}{Выполнение практической работы}

\section{Утилита sstabledump и ее установка}

Для начала необходимо скачать образ Cassandra и запустить контейнер.
Далее нужно зайти в консоль контейнера.

\img{img}{Консоль контейнера}

Скачаем пакет cassandra-tools
\



Создадим контейнер с первым узлом кассандры, где
\texttt{CASSANDRA\_CLUSTER\_NAME} можно задать своё.

Присоединяем к нему следующий узел, где
\texttt{CASSANDRA\_CLUSTER\_NAME} задаём такой же, как в первом узле.

Присоединяем к первому узлу следующий узел, где
\texttt{CASSANDRA\_CLUSTER\_NAME} задаём такой же, как в первом узле,
но меняем значение \texttt{CASSANDRA\_DC=datacenter1} на \texttt{datacenter2},
тем самым создаём новый узел в «как бы удалённом» датацентре:

Теперь у нас есть целый кластер кассандры, в котором все данные
синхронизированы, таким образом все модификации с бд, которые
будут проводиться будут применены ко всему кластеру и неважно с
какого узла эти самые изменения произошли (рис.~\ref{fig:docker}).

Вот docker-compose.yml фаил:

\lstinputlisting[language=yaml]{src/docker-compose.yml}

Запустим его:

\begin{image}
	\includegrph{img/img1.png}
	\caption{Запуск Docker compouse}
	\label{fig:docker}
\end{image}

Для того чтобы удостовериться, что весь кластер работает правильно
необходимо выполнить команду \texttt{docker exec -ti cas1 nodetool status}
(рис.~\ref{fig:status}).

\begin{image}
	\includegrph{img/img2.png}
	\caption{Вывод статуса узлов}
	\label{fig:status}
\end{image}

Теперь необходимо создать свою схему бд (в кассандре это
пространство ключей) для этого необходимо перейти в утилиту \texttt{cqlsh}
при помощи команды \texttt{docker exec -ti cas1 cqlsh} и выполнить скрипт,
где название кейспейса можно придумать самому (желательно номер
студенческого или другой идентификатор).

\begin{image}
	\includegrph{img/img3.png}
	\caption{Создание пространства ключей}
	\label{fig:keyspace}
\end{image}

Создадим базу данных и заполним ее значениями.
(рис.~\ref{fig:full}).

\begin{image}
	\includegrph{img/img4.png}
	\caption{Заполнение таблицы}
	\label{fig:full}
\end{image}

Теперь проверим, что на всех узлах есть созданные таблицы
(рис.~\ref{fig:check}).

\begin{image}
	\includegrph{img/img5.png}
	\caption{Содержимое пространства ключей на разных узлах}
	\label{fig:check}
\end{image}

\clearpage

Модель базы данных из прошлого семестра (рис.~\ref{fig:past}).

\begin{image}
	\includegrph{img/img6.png}
	\caption{Модель базы данных из прошлого семестра}
	\label{fig:past}
\end{image}

\section*{\LARGE Вывод}
\addcontentsline{toc}{section}{Вывод}
В результате практической работы была изучена работа
с распределенной бд, используя Apache Cassandra.\par
Были созданы докер контейнеры прдеставляющие разные узлы.
И базе данных созданы таблицы.