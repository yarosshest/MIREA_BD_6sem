\graphicspath{{/img}} % path to graphics

\section*{\LARGE Введение}
\addcontentsline{toc}{section}{Введение}

Распределенные базы данных можно широко классифицировать на
однородные и гетерогенные среды распределенных баз данных, каждая из
которых имеет дополнительные подразделения.

Свойства однородных распределенных баз данных:
\begin{itemize}
	\item На сайтах используется очень похожее программное обеспечение.
	\item Сайты используют идентичные СУБД или СУБД одного и того же
	производителя.
	\item Каждый сайт знает обо всех других сайтах и взаимодействует с
	другими сайтами для обработки пользовательских запросов.
	\item Доступ к базе данных осуществляется через единый интерфейс, как
	если бы это была одна база данных.
\end{itemize}

Типы однородной распределенной базы данных:
\begin{itemize}
	\item Автономный — каждая база данных независима и функционирует
	самостоятельно. Они интегрированы управляющим приложением и
	используют передачу сообщений для обмена обновлениями данных.
	\item Неавтономный — данные распределяются по однородным узлам, а
	центральная или главная СУБД координирует обновления данных по
	сайтам.
\end{itemize}

Свойства гетерогенных распределенных баз данных:
\begin{itemize}
	\item Система может состоять из множества СУБД, таких как реляционная,
	сетевая, иерархическая или объектно-ориентированная.
	\item Обработка запросов является сложной из-за разнородных схем.
	\item Обработка транзакций является сложной из-за различий в
	программном обеспечении.
	\item Сайт может не знать о других сайтах, поэтому сотрудничество при
	обработке пользовательских запросов ограничено.
\end{itemize}

Типы гетерогенных распределенных баз данных:
\begin{itemize}
	\item Федеративные — гетерогенные системы баз данных независимы по
	своей природе и объединены вместе, так что они функционируют как
	единая система баз данных.
	\item Без федерации — в системах баз данных используется центральный
	координационный модуль, через который осуществляется доступ к
	базам данных.
\end{itemize}

Кластер --- это группа узлов, объединённых каналами связи, представляющая с точки зрения пользователя единый ресурс для
хранения данных.
Установка Cassandra связана с большим количеством ошибок совместимости, поэтому целесообразнее установить и запустить
Docker-контейнер.

\textbf{Цель работы} --- cоздание кластера Apache Cassandra.

\clearpage

\section*{\LARGE Выполнение практической работы}
\addcontentsline{toc}{section}{Выполнение практической работы}

\section{Создание Docker compouse}

Создадим контейнер с первым узлом кассандры, где
\texttt{CASSANDRA\_CLUSTER\_NAME} можно задать своё.

Присоединяем к нему следующий узел, где
\texttt{CASSANDRA\_CLUSTER\_NAME} задаём такой же, как в первом узле.

Присоединяем к первому узлу следующий узел, где
\texttt{CASSANDRA\_CLUSTER\_NAME} задаём такой же, как в первом узле,
но меняем значение \texttt{CASSANDRA\_DC=datacenter1} на \texttt{datacenter2},
тем самым создаём новый узел в «как бы удалённом» датацентре:

Теперь у нас есть целый кластер кассандры, в котором все данные
синхронизированы, таким образом все модификации с бд, которые
будут проводиться будут применены ко всему кластеру и неважно с
какого узла эти самые изменения произошли (рис.~\ref{fig:docker}).

Вот docker-compose.yml фаил:

\lstinputlisting[language=yaml]{src/docker-compose.yml}

Запустим его:

\begin{image}
	\includegrph{img/img1.png}
	\caption{Запуск Docker compouse}
	\label{fig:docker}
\end{image}

Для того чтобы удостовериться, что весь кластер работает правильно
необходимо выполнить команду \texttt{docker exec -ti cas1 nodetool status}
(рис.~\ref{fig:status}).

\begin{image}
	\includegrph{img/img2.png}
	\caption{Вывод статуса узлов}
	\label{fig:status}
\end{image}

Теперь необходимо создать свою схему бд (в кассандре это
пространство ключей) для этого необходимо перейти в утилиту \texttt{cqlsh}
при помощи команды \texttt{docker exec -ti cas1 cqlsh} и выполнить скрипт,
где название кейспейса можно придумать самому (желательно номер
студенческого или другой идентификатор).

\begin{image}
	\includegrph{img/img3.png}
	\caption{Создание пространства ключей}
	\label{fig:keyspace}
\end{image}

Создадим базу данных и заполним ее значениями.
(рис.~\ref{fig:full}).

\begin{image}
	\includegrph{img/img4.png}
	\caption{Заполнение таблицы}
	\label{fig:full}
\end{image}

Теперь проверим, что на всех узлах есть созданные таблицы
(рис.~\ref{fig:check}).

\begin{image}
	\includegrph{img/img5.png}
	\caption{Содержимое пространства ключей на разных узлах}
	\label{fig:check}
\end{image}

\clearpage

Модель базы данных из прошлого семестра (рис.~\ref{fig:past}).

\begin{image}
	\includegrph{img/img6.png}
	\caption{Модель базы данных из прошлого семестра}
	\label{fig:past}
\end{image}

\section*{\LARGE Вывод}
\addcontentsline{toc}{section}{Вывод}
В результате практической работы была изучена работа
с распределенной бд, используя Apache Cassandra.\par
Были созданы докер контейнеры прдеставляющие разные узлы.
И базе данных созданы таблицы.