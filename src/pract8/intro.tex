\section*{\LARGE Введение}
\addcontentsline{toc}{section}{Введение}

Данная практическая работа направлена на закрепление полученных навыков
в области разработки программного обеспечения и баз данных.
В рамках этой работы студентам предлагается создать веб-приложение,
предназначенное для управления расписанием в высших учебных заведениях.

\begin{itemize}
    \item Тема проекта: Расписание в ВУЗе
    \item Вид приложения: Веб-интерфейс
    \item СУБД: MongoDB
    \item Требования к БД:
    \begin{itemize}
        \item Необходим кластер, состоящий минимум из 3 узлов
        \item Для создания необходимо использовать Docker
        \item Не менее 5 таблиц в базе данных
        \item Не менее 1 связи Один-Ко-Многим
        \item Не менее 1 связи Один-К-Одному
        \item Не менее 1 связи Один-К-Одному
    \end{itemize}
    \item Требования к функционалу системы (для каждой таблицы):
    \begin{itemize}
        \item Отображение всех документов collection
        \item Добавление документа
        \item Удаление документа
        \item Поиск документа по какому-либо критерию
        \item Сортировка документов
    \end{itemize}
\end{itemize}

\textbf{Требования к отчету}:
В результате вам необходимо предоставить скриншоты настройки и
запуска кластера, подключения к базе данных в коде, методы, реализующие
запросы к базе данных, а также скриншоты работы приложения.
\textbf{Целью} работы является не только создание работающего приложения,
но и демонстрация понимания принципов работы
с базами данных
\clearpage
