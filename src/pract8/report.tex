\graphicspath{{img}} % path to graphics
\section*{\LARGE Выполнение практической работы}
\addcontentsline{toc}{section}{Выполнение практической работы}

\section{Docker-compose}
Для обеспечения работы базы данных в контейнеризованной среде был разработан файл \texttt{docker-compose.yml}.
Этот файл представляет собой описание сервисов и их конфигураций,
которые будут развернуты с помощью Docker Compose.\par
В соответствии с требованиями, изложенными во введении,
в файле \texttt{docker-compose.yml} определены следующие сервисы:


\lstinputlisting[language=yaml]{code//docker-compose.yml}


Команда для запуска контейнеров на основе файла \texttt{docker-compose.yml}:

\begin{lstlisting}[language=bash]
docker-compose up -d --build
\end{lstlisting}


\section{Проетирования таблиц}
Была спроектирована архитектура базы данных, с помощью диаграммы классов

\begin{image}
    \includegrph{diagramm.drawio}
    \caption{Архитектура базы данных}
    \label{fig:architec}
\end{image}

\section{Реализация работы с колейций}
Для работы с коллекциями был создан интерфейс Collection

\lstinputlisting[language=python]{code//Collection.py}

Для данного интерфейса были реализованны реализации коллекций.

\lstinputlisting[language=python]{code//Address.py}
\lstinputlisting[language=python]{code//Clients.py}
\lstinputlisting[language=python]{code//Furniture.py}
\lstinputlisting[language=python]{code//FurnitureTypes.py}
\lstinputlisting[language=python]{code//Order.py}
\clearpage

\section{Работа приложения в действии}

В данной главе представлены скриншоты работы приложения,созданного на основе разработанных моделей и функционала.

Вывод подсказки:

\begin{image}
    \includegrph{img}
    \caption{Вывод подсказки}
    \label{fig:help}
\end{image}

Вывод коллекции:

\begin{image}
    \includegrph{img_1}
    \caption{Вывод коллекции}
    \label{fig:show}
\end{image}

\clearpage

Добавление документа:

\begin{image}
    \includegrph{img_2}
    \caption{Добавление документа}
    \label{fig:new}
\end{image}

\clearpage
Удаление документа по критерию:

\begin{image}
    \includegrph{img_3}
    \caption{Удаление документа}
    \label{fig:del}
\end{image}

Поиск документов по критерию с сортировкой:

\begin{image}
    \includegrph{img_4}
    \caption{Поиск документов}
    \label{fig:showWith}
\end{image}
\clearpage

Вывод коллекции с сортировкой:

\begin{image}
    \includegrph{img_5}
    \caption{Вывод коллекции с сортировкой}
    \label{fig:showSort}
\end{image}

\clearpage
Данные в mongo:
\begin{image}
    \includegrph{img_6}
    \caption{Данные в mongo}
    \label{fig:inmongo}
\end{image}

\section*{\LARGE Вывод}
\addcontentsline{toc}{section}{Вывод}

В результате выполнения данной практической работы было создано
приложение для управления производством мебели.
Для разработки приложения был выбран язык программирования Python.\par
Исходный код проекта доступен на GitHub по следующей ссылке:
\url{https://github.com/yarosshest/MIREA_6_sem_BD_con/tree/master}.