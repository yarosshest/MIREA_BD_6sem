\graphicspath{{img}} % path to graphics

\clearpage
\section*{\LARGE Выполнение практической работы}
\addcontentsline{toc}{section}{Выполнение практической работы}

\section{Установка MongoDB}

Прежде чем понять, как использовать шардирование в MongoDB,
необходимо изучить его базовый синтаксис.
Для использования mongoDB его можно нативно установить на свою
систему или использовать docker-образ. В данной практической работе будет
использован compose файл.

\begin{lstlisting}[language=bash]
version: '3.9'
services:
mongodb:
image: bitnami/mongodb
ports:
	- 27017:27017
volumes:
	- ~/apps/mongo:/data/db
environment:
	- MONGO_INITDB_ROOT_USERNAME=admin
	- MONGO_INITDB_ROOT_PASSWORD=admin
\end{lstlisting}

\clearpage
\section{Начало работы с Mongo}

Чтобы использовать MongoDB, выполните следующую команду.

\begin{lstlisting}[language=bash]
mongosh
\end{lstlisting}

Это подключит вас к работающему экземпляру MongoDB.
Чтобы получить список команд, введите \texttt{db.help()} в клиенте
MongoDB. Это даст вам список команд.\par
Чтобы получить статистику о сервере MongoDB, введите
команду \texttt{db.stats()} в клиенте MongoDB. Это покажет имя базы данных,
количество коллекций и документов в базе данных.

\begin{image}
	\includegrph{img}
	\caption{Mongosh}
	\label{fig:mongosh}
\end{image}

\clearpage
\section{Создание базы данных}

Для создания базы данных используется ключевое слово use.\par
Команда создаст новую базу данных, если она не существует, в противном
случае она вернет существующую базу данных.\par
Чтобы проверить текущую базу данных, используйте команду db.\par
Если вы хотите проверить список своих баз данных, используйте
команду show dbs.\par
Созданная вами база данных (mydb) отсутствует в списке. Для
отображения базы данных необходимо вставить в нее хотя бы один документ.\par
В MongoDB база данных по умолчанию “test”. Если вы не создавали базу
данных, коллекции будут храниться в этой базе данных.\par

\begin{image}
	\includegrph{img_1}
	\caption{Создание базы данных}
	\label{fig:create:db}
\end{image}

\clearpage
\section{Удаление базы данных}

Команда MongoDB db.dropDatabase() используется для удаления
существующей базы данных. Основной синтаксис команды следующий.\par
Это приведет к удалению выбранной базы данных. Если вы не выбрали
какую-либо базу данных, она удалит тестовую базу данных по умолчанию.
Сначала проверьте список доступных баз данных с помощью
команды show dbs.\par
Если вы хотите удалить новую базу данных <mydb>, команда
dropDatabase() будет выглядеть следующим образом.\par
Теперь проверьте список баз данных.

\begin{image}
	\includegrph{img_2}
	\caption{Удаление базы данных}
	\label{fig:drop:db}
\end{image}

\clearpage
\section{Создание коллекций}

\textbf{db.createCollection (имя, параметры)} используется для создания
коллекции.
В команде name --- это имя создаваемой коллекции. Options — это
документ, который используется для указания конфигурации коллекции.

\begin{image}
	\includegrph{img_3}
	\caption{Создание коллекций}
	\label{fig:create:collection}
\end{image}

\clearpage
\section{Удаление коллекций}

\texttt{db.collection.drop()} MongoDB используется для удаления коллекции из
базы данных.

\begin{image}
	\includegrph{img_4}
	\caption{Создание коллекций}
	\label{fig:create:collection}
\end{image}

\clearpage
\section{Типы данных}
MongoDB поддерживает множество типов данных.
Некоторые из них:
String — это наиболее часто используемый тип данных для хранения
данных. Строка в MongoDB должна быть допустимой в кодировке UTF-8.\par
Integer — этот тип используется для хранения числового значения. Целое
число может быть 32-битным или 64-битным в зависимости от вашего сервера.\par
Boolean — этот тип используется для хранения логического (истинного /
ложного) значения.
Double —тип используется для хранения значений с плавающей запятой.\par
Массивы — этот тип используется для хранения массивов, списков или
нескольких значений в одном ключе.\par
Отметка времени — ctimestamp. Это может быть удобно для
фиксирования, когда документ был изменен или добавлен.\par
Object — этот тип данных используется для встроенных документов.\par
Null — этот тип используется для хранения нулевого значения.\par
Symbol — этот тип данных используется идентично строке; однако
обычно он зарезервирован для языков, использующих определенный тип
символов.\par
Date — этот тип данных используется для хранения текущей даты или
времени в формате времени UNIX. Вы можете указать свое собственное время
даты, создав объект Date и передав в него день, месяц, год.\par
Object ID— этот тип данных используется для хранения идентификатора
документа.\par
Binary data — этот тип данных используется для хранения двоичных
данных.\par
Code — этот тип данных используется для хранения кода JavaScript в
документе.\par
Регулярное выражение — этот тип данных используется для хранения
регулярного выражения.\par

\clearpage
\section{Вставка документа}

Чтобы вставить документ в коллекцию MongoDB, вам нужно
использовать метод MongoDB insert() или save().

\begin{image}
	\includegrph{img_5}
	\caption{Вставка документа}
	\label{fig:insert}
\end{image}

\clearpage
\section{Запрос данных}

Чтобы запросить данные из коллекции MongoDB, вам нужно
использовать метод find() MongoDB.\par
Метод find() отобразит все документы в неструктурированном виде.
Чтобы отобразить результаты в отформатированном виде, вы можете
использовать метод pretty().

\begin{image}
	\includegrph{img_6}
	\caption{Запрос данных}
	\label{fig:find}
\end{image}

\begin{image}
	\includegrph{img_7}
	\caption{Запрос данных findOne}
	\label{fig:find:one}
\end{image}

\begin{image}
	\includegrph{img_8}
	\caption{Запрос данных с условием}
	\label{fig:find:like}
\end{image}

\clearpage
\section{Обновление документов}

Методы MongoDB update() и save() используются для обновления
документа в коллекции. Метод update() обновляет значения в существующем
документе, а метод save() заменяет существующий документ документом,
переданным в методе save().

\begin{image}
	\includegrph{img_9}
	\caption{Обновление документов}
	\label{fig:update}
\end{image}

\clearpage
\section{Удаление документа}

Метод remove() используется для удаления документа из
коллекции. Метод remove() принимает два параметра. Один из них — критерии
удаления, а второй — флаг justOne.

\begin{itemize}
	\item критерии удаления — (необязательно) критерии в соответствии с
	которыми будут удалены документы.
	\item justOne — (необязательно), если установлено значение true или 1,
	удалит только один документ.
\end{itemize}

\begin{image}
	\includegrph{img_10}
	\caption{Удаление документа}
	\label{fig:remove}
\end{image}

\clearpage
\section{Проекция}

В MongoDB проекция означает выбор только необходимых данных, а не
выбор всех данных документа. Если в документе 5 полей и вам нужно показать
только 3, то выберите из них только 3 поля.\par
Метод find() принимает второй необязательный параметр, который
представляет собой список полей, которые вы хотите получить. В MongoDB при
выполнении метода find() отображаются все поля документа. Чтобы ограничить
это, вам нужно установить список полей со значением 1 или 0. 1 используется
для отображения поля, а 0 используется для скрытия полей.

\begin{image}
	\includegrph{img_11}
	\caption{Проекция}
	\label{fig:select}
\end{image}

\clearpage
\section{Ограничение количества записей}

Чтобы ограничить записи в MongoDB, вам нужно использовать метод
limit(). Метод принимает один аргумент целочисленного типа, который
представляет собой количество документов, которые вы хотите отобразить.

\begin{image}
	\includegrph{img_12}
	\caption{Ограничение количества записей}
	\label{fig:limit}
\end{image}

\clearpage
\section{Cортировка записей}

Чтобы отсортировать документы в MongoDB, вам нужно использовать
метод sort(). Метод принимает документ, содержащий список полей вместе с
порядком их сортировки. Для указания порядка сортировки используются 1 и -1.
1 используется для возрастания, а -1 используется для убывания

\begin{image}
	\includegrph{img_13}
	\caption{Cортировка записей}
	\label{fig:sort}
\end{image}

\clearpage
\section{Индексирование}

Индексы поддерживают эффективное выполнение запросов. Без индексов
MongoDB должна сканировать каждый документ коллекции, чтобы выбрать те
документы, которые соответствуют оператору запроса. Это сканирование
крайне неэффективно и требует от MongoDB обработки большого объема
данных.\par
Индексы — это специальные структуры данных, которые хранят
небольшую часть набора данных в удобной для просмотра форме. В индексе
хранится значение определенного поля или набора полей, упорядоченных по
значению поля, указанному в индексе.

\begin{image}
	\includegrph{img_14}
	\caption{Индексирование}
	\label{fig:index}
\end{image}

\clearpage

\section*{\LARGE Вывод}
\addcontentsline{toc}{section}{Вывод}
В данной практической работе мы изучили СУБД Mongo.
MongoDB --- это база данных
документов, в которой коллекции могут содержать разные документы.
Количество полей, содержание и размер документа могут отличаться от одного
документа к другому.