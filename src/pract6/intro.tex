\section*{\LARGE Введение}
\addcontentsline{toc}{section}{Введение}

В динамично растущих системах, объемы данных, как правило, быстро
увеличиваются и рано или поздно можно столкнуться с проблемой, когда
текущих ресурсов машины будет не хватать для нормальной работы.
Для решения этой проблемы применяют горизонтальное и вертикальное
масштабирование. Шардирование является частным случаем горизонтального
масштабирования. Суть его в разделении базы данных на отдельные части по
определенному правилу так, чтобы каждую из них можно было вынести на
отдельный сервер. Однако реплицирование данных является обязательным
требованием для шардирования в Mongo. Сервер конфигурации и каждый шард
представляют из себя так называемый Реплика-сет (группа экземпляров,
хранящих одинаковые наборы данных).